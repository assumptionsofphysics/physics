\documentclass[aps,pra,10pt,twocolumn,floatfix,nofootinbib]{revtex4-1}

\usepackage{bbm}
\usepackage{amsmath}
\usepackage{amssymb}
\usepackage{graphicx}
\usepackage{amsfonts}
\usepackage{amsthm}

\newtheorem{thm}{Theorem}[section]
\newtheorem{cor}[thm]{Corollary}
\newtheorem{lem}[thm]{Lemma}
\newtheorem{prop}[thm]{Proposition}

\theoremstyle{definition}
\newtheorem{defn}[thm]{Definition}
\newtheorem*{assump1}{Classical assumption}
\newtheorem*{assump2}{Determinism and Reversibility assumption}

\begin{document}

\title{Generalized Hamiltonian mechanics}
\author{Gabriele Carcassi}
\affiliation{University of Michigan, Ann Arbor, MI 48109}
\email{carcassi@umich.edu}
\date{June 16, 2014}

\begin{abstract}
Generalizing Hamiltonian to any coordinate variables.\end{abstract}

\maketitle

Use roman letters for phase space indexes.

\begin{align*}
&S^a=d_ta \\
&S_a = \partial_aH \\
&S_b = S^a \omega_{a, b} \\
\end{align*}

\begin{align*}
&a'=f(a) \\
&S^{a'}=d_ta'=\partial_aa'd_ta=\partial_aa'S^a \\
&S_{a'} = \partial_{a'}H = \partial_{a'}a\partial_{a}H = \partial_{a'}aS_a \\
&S_{b'} = S^{a'} \omega_{a', b'} = \partial_ca'S^c \omega_{a', b'} = \partial_{b'}d \, S_d \\
&S_d = \partial_db' \partial_ca' \omega_{a', b'} S^c = \omega_{c, d} S^c \\
\end{align*}

Example
\begin{align*}
&H(q,p) = \frac{p^2}{2m} + \frac{kq^2}{2}  \\
&\omega_{a, b} =  \left[
  \begin{array}{cc}
    0 & 1 \\
    -1 & 0 \\
  \end{array}
\right] \\
&\omega_{q,p} = 1 \\
&\omega_{p,q} = -1 \\
&S^q \omega_{q, p} = S_p = \partial_p H \\
&d_tq = \frac{p}{m} \\
&S^p \omega_{p, q} = S_q = \partial_q H \\
&d_tp (-1) = kq \\
&d_tq = \frac{p}{m} \\
&d_tp = -kq \\
\end{align*}

\begin{align*}
&p' = p &q' = q^3 \\
&p = p' &q = q'^{1/3} \\
&\partial_{p'}p=1 \\
&\partial_{q'}q=\frac{1}{3}q'^{-2/3} \\
&\partial_{p}p'=1 \\
&\partial_{q}q'=3q^2=3q'^{2/3} \\
&H(q',p') = \frac{p'^2}{2m} + \frac{kq'^{2/3}}{2}  \\
&\omega_{q',p'} = \partial_{q'}a \partial_{p'}b \omega_{a,b} = \partial_{q'}q \partial_{p'}p \omega_{q,p} \\
&\omega_{q',p'} = \frac{1}{3}q'^{-2/3} \\
&\omega_{p',q'} = \partial_{p'}a \partial_{q'}b \omega_{a,b} = \partial_{p'}p \partial_{q'}q \omega{p,q} \\
&\omega_{p,q} = -\frac{1}{3}q'^{-2/3} \\
&\omega_{a', b'} =  \left[
  \begin{array}{cc}
    0 & \frac{1}{3}q'^{-2/3} \\
    -\frac{1}{3}q'^{-2/3} & 0 \\
  \end{array}
\right] \\
&S^{q'} \omega_{q', p'} = S_{p'} = \partial_{p'} H \\
&d_tq' \frac{1}{3}q'^{-2/3} = \frac{p}{m} \\
&S^{p'} \omega_{p', q'} = S_{q'} = \partial_{q'} H \\
&d_tp (-\frac{1}{3}q'^{-2/3}) = \frac{k}{2} \frac{2}{3} q'^{-1/3} \\
&d_tq' = \frac{3p'q'^{3/2}}{m} \\
&d_tp' = - k q'^{1/3} \\
\end{align*}
Should be:
\begin{align*}
&d_tp'=d_pp'd_tp=d_tp=-kq=-kq'^{1/3} \\
&d_tq'=d_qq'd_tq=3q^2d_tq=3q^2p/m=\frac{3p'q'^{2/3}}{m}
\end{align*}


\begin{align*}
v^{\alpha} \omega_{\alpha, \beta} w^{\beta} &= v'^{\alpha} \omega_{\alpha, \beta} w'^{\beta}  \\
&= (v^{\alpha} + \partial_{\gamma} S^{\alpha} dt v^{\gamma}) \omega_{\alpha, \beta} ( w^{\beta} + \partial_{\delta} S^{\beta} w^{\delta} dt) \\
&= v^{\alpha} \omega_{\alpha, \beta} w^{\beta} + (\partial_{\gamma} S^{\alpha} v^{\gamma} \omega_{\alpha, \beta} w^{\beta} \\
 &+ v^{\alpha} \omega_{\alpha, \beta} \partial_{\delta} S^{\beta} w^{\delta}) dt + O(dt^2)
\end{align*}
\begin{align*}
v^{\gamma} w^{\beta} \partial_{\gamma} S_{\beta} - v^{\alpha} w^{\delta} \partial_{\delta} S_{\alpha} = 0
\end{align*}
\begin{align*}
\partial_{\alpha} S_{\beta} - \partial_{\beta} S_{\alpha} &= curl(S_{\alpha}) = 0 \\
S_{\alpha} &= \partial_{\alpha}H
\end{align*}

\begin{align*}
d_{t}q &= \partial_{p} H \\
d_{t}p &= - \partial_{q} H
\end{align*}

\begin{align*}
\omega_{\alpha, \beta} =  \left[
  \begin{array}{cc}
    1 & 0 \\
    0 & 1 \\
  \end{array}
\right] \otimes \left[
  \begin{array}{cc}
    0 & 1 \\
    -1 & 0 \\
  \end{array}
\right] =
\left[
  \begin{array}{cccc}
    0 & 1 & 0 & 0 \\
    -1 & 0 & 0 & 0 \\
    0 & 0 & 0 & 1 \\
    0 & 0 & -1 & 0 \\
  \end{array}
\right] \\
\end{align*}

\begin{align*}
d_{t}q^i &= \partial_{p^i} H \\
d_{t}p^i &= - \partial_{q^i} H
\end{align*}

\begin{align*}
\omega_{\alpha, \beta} =  \left[
  \begin{array}{cc}
    -1 & 0 \\
    0 & 1 \\
  \end{array}
\right] \otimes \left[
  \begin{array}{cc}
    0 & 1 \\
    -1 & 0 \\
  \end{array}
\right]
= \left[
  \begin{array}{cccc}
    0 & -1 & 0 & 0 \\
    1 & 0 & 0 & 0 \\
    0 & 0 & 0 & 1 \\
    0 & 0 & -1 & 0 \\
  \end{array}
\right] \\
\end{align*}


\begin{thebibliography}{0}

\bibitem{Jaynes} Jaynes, E. T.: ``Information theory and statistical mechanics'', (1963)
\bibitem{Shannon} Shannon, C. E.: ``A mathematical theory of communications'', The Bell System Technical Journal, Vol. 27, pp. 379–423, 623–656, (1948)
\bibitem{classical_dynamics} J. V. Jose', E. J. Saletan: ``Classical Dynamics'', Cambridge University Press, (1998)
\bibitem{Gromov} Gromov, M. L.: ``Pseudo holomorphic curves in symplectic manifolds''. Inventiones Mathematicae 82: 307–347, (1985)
\bibitem{deGosson} de Gosson, M. A.: ``The symplectic camel and the uncertainty principle: the tip of an iceberg?'', Foundation of Physics 39, pp. 194–214, (2009)
\bibitem{Stewart} Stewart, I.: ``The symplectic camel'', Nature 329(6134), 17–18 (1987)
\bibitem{Lanczos} Lanczos, C.: ``The variational principles of mechanics'', University of Toronto Press (1949)
\bibitem{Synge} Synge, J. L.: Encyclopedia of Physics Vol 3/1, Springer (1960)
\bibitem{Struckmeier} Struckmeier, J.: ``Hamiltonian dynamics on the symplectic extended phase space for autonomous and non-autonomous systems'', J. Phys. A: Math. Gen 38, 1257-1278, (2005)

\end{thebibliography}

\end{document}
