\newif\ifjournal
%\journaltrue
\journalfalse

\ifjournal
	\documentclass[smallextended]{svjour3} 
\else 
	\documentclass[aps,pra,10pt,twocolumn,floatfix,nofootinbib]{revtex4-1}
\fi

\usepackage{amsmath}
\usepackage{amssymb}
\usepackage{graphicx}
\usepackage{amsfonts}
\usepackage{dutchcal}
\usepackage{braket}
\usepackage{enumitem}

\usepackage{tikz}
\usetikzlibrary{calc}
\usepackage{calculator}
\usepackage{standalone}

\numberwithin{equation}{section}

\ifjournal
	\spnewtheorem{assump}{Assumption}{\bf}{\it}
	\renewcommand{\theassump}{\Roman{assump}}
	\spnewtheorem{prop}[equation]{Proposition}{\bf}{\rm}
	\spnewtheorem{thrm}[equation]{Theorem}{\bf}{\it}

	\newenvironment{rationale}{\emph{Rationale}.}{\hfill\(\qed\)}
	\newenvironment{justification}{\emph{Justification}.}{\hfill\(\qed\)}
	\renewenvironment{proof}{\emph{Proof}.}{\hfill\(\qed\)}
\else
	% Theorem definitions using amsthm

	\usepackage{amsthm}

	\renewcommand\thesubsection{\thesection.\Alph{subsection}}
	\renewcommand{\theequation}{\thesection.\arabic{equation}}

	\newtheorem{assump}{Assumption}
	\renewcommand*{\theassump}{\Roman{assump}}
	\newtheorem{thrm}[equation]{Theorem}

	\theoremstyle{definition}
	\newtheorem{prop}[equation]{Proposition}

	\newenvironment{rationale}{\emph{Rationale}.}{\qed}
	\newenvironment{justification}{\emph{Justification}.}{\qed}
	\renewenvironment{proof}{\emph{Proof}.}{\qed}

\fi


%TODO: decide on identity function typesetting
\newcommand{\id}{\textrm{id}}

\newcommand{\journal}[1]{\ifjournal#1\fi}
\newcommand{\arxiv}[1]{\ifjournal\else#1\fi}

\renewcommand{\descriptionlabel}[1]{%
	\hspace\labelsep \upshape\bfseries #1:%
}

\begin{document}

\title{DRAFT \\ Hamiltonian mechanics is conservation of information}
\author{Gabriele Carcassi, Christine A. Aidala}

\ifjournal
	\institute{G. Carcassi \at
		University of Michigan, Ann Arbor, MI 48109 \\
		\email{carcassi@umich.edu}           %  \\
	}
\else
	\affiliation{University of Michigan, Ann Arbor, MI 48109}
\fi

%\ifjournal
%	% As the page margins are so large, the full title does not fit
%	\titlerunning{From physical principles to classical and quantum particle mechanics}
%\fi

\date{\today}

% Journal format has this before the abstract
\journal{\maketitle}
	
\begin{abstract}
\textbf{This manuscript is a work in progress.} Ideas are constantly reshaped to find more precise and elegant arguments. It is provided as is to stimulate discussion.  \textbf{Make sure you have the latest version from http://assumptionsofphysics.org}

In this work we show that the canonical transformations of classical Hamiltonian mechanics are exactly the transformations that preserve information.
\end{abstract}

\arxiv{\maketitle}

\section{Introduction}

\section{Something}

\begin{prop}
	Transported coordinates are canonical. Distributions are function of the point and are transported.
\end{prop}

$\mathcal{M} = T^*\mathcal{Q}$. $(T^*\mathcal{Q}, \omega)$, $(q^i, p_j)$ set of canonical coordinates, $F: T^*\mathcal{Q} \rightarrow T^*\mathcal{Q}$ such that $\omega = F^* \omega$ where $F^*$ is the pullback. $\hat{q}^i = q^i \circ F^{-1}$ and $\hat{p}_i = p_i \circ F^{-1}$.

$F^* (\sum d\hat{q}^i \wedge d\hat{p}_i) =\sum F^* (d\hat{q}^i \wedge d\hat{p}_i) =  \sum (F^* d\hat{q}^i)\wedge (F^* d\hat{p}_i) = \sum d(\hat{q}^i \circ F)\wedge d(\hat{p}_i \circ F) = \sum d(q^i \circ F^{-1} \circ F)\wedge d(p_i \circ F^{-1} \circ F) = \sum dq^i\wedge dp_i = \omega = F^* \omega$. $\omega =  \sum d\hat{q}^i \wedge d\hat{p}_i$. $(\hat{q}^i, \hat{p}_j)$ is a set of canonical coordinates.

$\rho(q^i, p_j) : \mathbb{R}^{2n} \rightarrow \mathbb{R}$ integrable and differentiable. $\rho (q^i, p_j) \, dq^n \wedge dp_n = \rho (q^i, p_j) \, \frac{\omega^n}{n!} = \rho (q^i(\hat{q}^i, \hat{p}_j), p_j(\hat{q}^i, \hat{p}_j)) \, d\hat{q}^n \wedge d\hat{p}_n$. $\rho$, defined on the symplectic structure, transforms like a function. We can write $\rho : (T^*\mathcal{Q}, \omega) \rightarrow \mathbb{R}$.

$\hat{\rho} : (T^*\mathcal{Q}, \omega) \rightarrow \mathbb{R}$ such that $\hat{\rho} = \rho \circ F^{-1 }$. $\hat{\rho}(\hat{q}^i, \hat{p}_j) = \hat{\rho} \circ (\hat{q}^i, \hat{p}_j)^{-1} = \rho \circ F^{-1} \circ (q^i \circ F^{-1}, p_i \circ F^{-1}) ^ {-1}= \rho \circ F^{-1} \circ F \circ (q^i, p_j) ^ {-1} = \rho \circ (q^i, p_j) ^ {-1} = \rho (q^i, p_j)$.

\begin{prop}
	Canonical transformations transport marginal distributions and preserve independence.
\end{prop}

$\mathcal{Q} = \mathcal{Q}_1 \times \mathcal{Q}_2$. $(T^*\mathcal{Q}, \omega)= (T^*\mathcal{Q}_1, \omega_1) \times (T^*\mathcal{Q}_2, \omega_2)$, $(q^i, p_j)$ set of canonical coordinates divides into $(q_1^i, p_{1j})$ and $(q_2^i, p_{2j})$ canonical coordinates.

$\rho : (T^*\mathcal{Q}, \omega) \rightarrow \mathbb{R}$. $\frac{(\omega)^n}{n!} = \frac{(\omega_1 + \omega_2)^n}{n!} = {n \choose n_1} \frac{(\omega_1)^{n_1} \wedge (\omega_2)^{n_2}}{n!} = \frac{n!}{n_1!n_2!}\frac{(\omega_1)^{n_1} \wedge (\omega_2)^{n_2}}{n!} = \frac{(\omega_1)^{n_1}}{n_1!} \wedge \frac{(\omega_2)^{n_2}}{n_2!} $. $\int_{T^*\mathcal{Q_1}}\rho \frac{(\omega)^n}{n!} = \int_{T^*\mathcal{Q_1}}\rho \frac{(\omega_1)^{n_1}}{n_1!} \wedge \frac{(\omega_2)^{n_2}}{n_2!} = \frac{(\omega_2)^{n_2}}{n_2!} \wedge \int_{T^*\mathcal{Q_1}}\rho \frac{(\omega_1)^{n_1}}{n_1!}$. Let $\rho_2 : (T^*\mathcal{Q}_2, \omega) \rightarrow \mathbb{R}$ such that $\rho_2 = \int_{T^*\mathcal{Q_1}}\rho \frac{(\omega_1)^{n_1}}{n_1!}$. $\int_{T^*\mathcal{Q_1}}\rho \frac{(\omega)^n}{n!} = \rho_2 \frac{(\omega_2)^{n_2}}{n_2!}$. Similarly, $\int_{T^*\mathcal{Q_2}}\rho \frac{(\omega)^n}{n!} = \rho_1 \frac{(\omega_1)^{n_1}}{n_1!}$ with $\rho_1 : (T^*\mathcal{Q}_1, \omega) \rightarrow \mathbb{R}$.

$F(T^*\mathcal{Q}, \omega) = F((T^*\mathcal{Q}_1, \omega_1) \times (T^*\mathcal{Q}_2, \omega_2)) = F_1(T^*\mathcal{Q}_1, \omega_1) \times F_2(T^*\mathcal{Q}_2, \omega_2) = (\hat{\mathcal{M}}_1, \hat{\omega}_1) \times (\hat{\mathcal{M}}_2, \hat{\omega}_2)$.
Let $\hat{\rho}_1 : (\hat{\mathcal{M}}_1, \hat{\omega}_1) \rightarrow \mathbb{R}$ and $\hat{\rho}_2 : (\hat{\mathcal{M}}_2, \hat{\omega}_2) \rightarrow \mathbb{R}$ such that $\hat{\rho}_1 = \rho_1 \circ F^{-1 }$ and $\hat{\rho}_2 = \rho_2 \circ F^{-1 }$.  $\hat{\rho}_1(\hat{q}_1^i, \hat{p}_{1j})=\rho_1(q_1^i, p_{1j})$ and $\hat{\rho}_2(\hat{q}_2^i, \hat{p}_{2j})=\rho_2(q_2^i, p_{2j})$.

Let $\rho=\rho_1\rho_2$. $\hat{\rho}(\hat{q}^i, \hat{p}_j) =\rho(q^i, p_j)=\rho_1(q_1^i, p_{1j})\rho_2(q_2^i, p_{2j})=\hat{\rho}_1(\hat{q}_1^i, \hat{p}_{1j})\hat{\rho}_2(\hat{q}_2^i, \hat{p}_{2j})$. $\hat{\rho}=\hat{\rho}_1\hat{\rho}_2$.

Prove the opposite: from transport and independence of marginal distributions to canonical transformation.

Conjecture: the manifold for the marginal distributions is not always a cotangent bundle.


\begin{prop}
	Canonical transformations transport mutual information.
\end{prop}

\begin{thebibliography}{0}
	
\end{thebibliography}

\end{document}
