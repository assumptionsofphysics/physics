\documentclass[aps,pra,10pt,twocolumn,floatfix,nofootinbib]{revtex4-1}

\usepackage{bbm}
\usepackage{amsmath}
\usepackage{amssymb}
\usepackage{graphicx}
\usepackage{amsfonts}
\usepackage{amsthm}

\newtheorem{thm}{Theorem}[section]
\newtheorem{cor}[thm]{Corollary}
\newtheorem{lem}[thm]{Lemma}
\newtheorem{prop}[thm]{Proposition}

\theoremstyle{definition}
\newtheorem{defn}[thm]{Definition}
\newtheorem*{assump1}{Classical assumption}
\newtheorem*{assump2}{Determinism and Reversibility assumption}

\begin{document}

\title{From physical principles to classical Lagrangian mechanics}
\author{Gabriele Carcassi}
\affiliation{University of Michigan, Ann Arbor, MI 48109}
\email{carcassi@umich.edu}
\date{June 16, 2014}

\begin{abstract}
We derive the Langrangian formalism from the assumption that trajectories in space-time are the output of a deterministic and reversible process. \end{abstract}

\maketitle

\section{Introduction}


\section{Hamiltonian review}

\section{Kinematics}
Assume trajectory are deterministic and reversible. Have state.

How much state? Find that is two per degree of freedom. Position and momentum are only state variables.

Inertial frame

\section{Connect to Hamiltonian}
Let q=x

Find that p must be contra-variant. Must be monotonic. And a linear transformation of v. Introduce mass. Introduce gauge.

\section{Lagrangian}

From Hamiltonian conservation to Lagrangian. p monotonic means convex Hamiltonian: can use Legendre transform.

\section{Conclusion}

\begin{thebibliography}{0}

\bibitem{Jaynes} Jaynes, E. T.: ``Information theory and statistical mechanics'', (1963)
\bibitem{Shannon} Shannon, C. E.: ``A mathematical theory of communications'', The Bell System Technical Journal, Vol. 27, pp. 379–423, 623–656, (1948)
\bibitem{classical_dynamics} J. V. Jose', E. J. Saletan: ``Classical Dynamics'', Cambridge University Press, (1998)
\bibitem{Gromov} Gromov, M. L.: ``Pseudo holomorphic curves in symplectic manifolds''. Inventiones Mathematicae 82: 307–347, (1985)
\bibitem{deGosson} de Gosson, M. A.: ``The symplectic camel and the uncertainty principle: the tip of an iceberg?'', Foundation of Physics 39, pp. 194–214, (2009)
\bibitem{Stewart} Stewart, I.: ``The symplectic camel'', Nature 329(6134), 17–18 (1987)
\bibitem{Lanczos} Lanczos, C.: ``The variational principles of mechanics'', University of Toronto Press (1949)
\bibitem{Synge} Synge, J. L.: Encyclopedia of Physics Vol 3/1, Springer (1960)
\bibitem{Struckmeier} Struckmeier, J.: ``Hamiltonian dynamics on the symplectic extended phase space for autonomous and non-autonomous systems'', J. Phys. A: Math. Gen 38, 1257-1278, (2005)

\end{thebibliography}

\end{document}
