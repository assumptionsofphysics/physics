\documentclass[aps,pra,10pt,twocolumn,floatfix,nofootinbib]{revtex4-1}

\usepackage{bbm}
\usepackage{amsmath}
\usepackage{amssymb}
\usepackage{graphicx}
\usepackage{amsfonts}
\usepackage{amsthm}
\usepackage{dutchcal}

\newtheorem{assump}{Assumption}
\renewcommand*{\theassump}{\Roman{assump}}
\newtheorem{prop}{Proposition}[section]

\theoremstyle{definition}
\newtheorem{defn}[prop]{Definition}

\newenvironment{rationale}{\emph{Rationale}.}{\qed}
\newenvironment{justification}{\emph{Justification}.}{\qed}
\renewenvironment{proof}{\emph{Proof}.}{\qed}

\begin{document}

\title{From physical principles to ???}
\author{Gabriele Carcassi}
\affiliation{University of Michigan, Ann Arbor, MI 48109}
\email{carcassi@umich.edu}
\date{March 15, 2015}

\begin{abstract}
Deterministic/reversible assumption. Describe fundamental model of physics. System is always interacting with environment, both deterministically and non-deterministically. Deterministic means only depends on the state of the system. Set of states needs to be invariant under non-deterministic interaction: the non-deterministic interaction affects what are the states of the system. Deterministic/reversible evolution is necessary to even be able to talk about a system (trivial deterministic process).

Define: set of states, deterministic evolution as permutation, labels/state variables, cardinality of state variable, state id function (=1 for a particular state). Can we do for continuous variables as well? Can we find divergence free evolution for det/rev and independent variables?

Infinite reducibility assumption. Decomposable systems. Rule of composition. Inverse and null state to study changes. Set of transformations to increase and decrease the quantity in each state. Define magnitude of the system (probably needed to guarantee things converge in math?).

Define: set of states as a vector space, inner product.

Continuous quantities. Need for densities to be a function of the state (not state variable). Defined on infinitesimal interval. Define: degree of freedom, state cardinality, metric (T*Q).
\end{abstract}
\maketitle

\section{Introduction}

Classical particle mechanics is usually founded on Newton's laws. These, though, are insufficient to derive the full Lagrangian and Hamiltonian formalism, and usually other ad-hoc assumptions (e.g. conservative forces) are introduced. Special relativity is based on two principles (invariance of the speed of light and the principle of relativity), which lead to the Minkowskian nature of space-time but not to the equations of motion, which are a consistent reformulation of the non-relativistic ones. Quantum mechanics is simply founded on its mathematical formulation, leaving its physical significance open to debate. One can't come to the conclusion that, unfortunately, the landscape of fundamental physics is a piece-meal of concepts and idea, sometime apparently inconsistent with each other. Yet: the physical world is one, and our physical theories are simply models of such reality. If we clarify what those models represent, and on what assumptions are based, shouldn't we be able to have a coherent and more satisfying picture?

This work aims to re-organize the known elements and equations in a more consistent and comprehensive way, leading to better insight on why the fundamental concepts and laws are what they are. We will start with physical assumptions, that clarify the models we impose on the physical world, and give arguments on when those assumptions can be considered valid. We will justify our mathematical definitions, based on the formal properties of the objects of our discussion. And will, as a consequence, re-derive known results and theories. We will strive to do this in a way that is mathematically meaningful, philosophically consistent and mathematically precise.

TODO: high level view of the derivation

%We'll use concepts from different disciplines, such as set theory, differential geometry, relativity, Hamiltonian and Lagrangian mechanics, and we'll find interesting connections among them. We'll keep names and notation as consistent as possible to current use across the different disciplines. This may sometimes lead to some non sequitur as it will not be immediately clear why the new definitions are equivalent to the standard ones. These are typically resolved by subsequent derivation of the expected properties.

%No mathematical breakthrough should be expected: the goal, after all, is to derive the \emph{known} framework from a set of \emph{simple} definitions in the most \emph{obvious} way possible. No proof is longer than a couple of paragraphs, so the word \emph{theorem} is avoided in favor of \emph{proposition} and \emph{corollary}. The novel, and surprising, result is how so much can be derived from so little.

\section{About}

The work is organized into:
\begin{description}
  \item[Assumptions] these characterize the physical system we are studying and constitute the premise of our discussion. A \textbf{rationale} follows each assumption, which uses physical and sometimes philosophical arguments to motivate why (or why not) such an assumption makes sense (in a particular case).
  \item[Definitions] these encode into mathematical language the physical concepts we are studying. A \textbf{justification} follows each definition, which uses physical and mathematical arguments to explain why such a definition and its properties are necessary to describe a particular physical concept.
  \item[Propositions] these are statements that mathematically follow from definitions or other propositions. A mathematical \textbf{proof} follows each proposition. No mathematical breakthrough should be expected as we mostly use well known results from different fields. The goal is to show how those results make sense in light of the physical assumptions we start with.
\end{description}
This should allow you, the reader, to focus on the parts you are most interested in and skim the rest. The philosophically inclined may focus on the assumptions, their rationale and the definitions. The mathematically inclined may focus on the definitions, the propositions and their proofs. The scientist on the assumptions, the definitions and their justifications, and the propositions.

\section{Style objective}

The writing style should have the following goals:
\begin{description}
  \item[Skip math] one should be able to skip the mathematical definitions/propositions and still get a coherent narrative.
  \item[Well-defined physical objects] each mathematical objects need to be physically well-defined. It must shown to exist and be unique.
\end{description}
Non-goals:
\begin{description}
  \item[Can't skip physics in justifications] justifications of mathematical definitions will inherently be rooted in the preceding physics discussion.
\end{description}
Notation:
\begin{description}
  \item[state variable and space] state space (cont/discr), state variable (cont/discr), possibility (cont/discr), range of possibility (cont/discr), set of possibilities (cont/discr)
\end{description}

\section{Determinism and reversibility}

%TODO: it would be nice to have "examples" always refer to the same cases, and that they pertain to different type of physics. E.g. trajectory, gravity, thermodynamics, fluid dynamics, biology, electromagnetism...

We start by fixing a \emph{physical system} we want to study, that is something we can interact with and perform measurements on (e.g. a cat, a planet, water). We call \emph{environment} everything else. We set what particular aspect we want to study (e.g. the trajectory when falling, the motion around a star, its flow in a pipe). We call \emph{state} a particular configuration in time of the portion under study. Since the state does not, in general, exhaust the description of the system, a part remains \emph{unstated}, and as such we'll call it, for lack of a better word. With this in mind, we introduce the following:

\begin{assump}[Determinism and reversibility]
The state of the physical system under study undergoes deterministic and reversible evolution.
\end{assump}

By deterministic (and reversible) we mean that future (and past) states are uniquely identified by the present state. We can then think of the state of the system as the part that interacts deterministically and reversibly with the environment (and itself) while the unstated part as interacting non-deterministically with the environment (and itself). We call this setup the \emph{fundamental model of physics}, as it is the simplest model that still captures the crucial aspects of studying and writing laws for a physical system.

Much of the focus later will be given to the state under deterministic and reversible evolution. However, the non-deterministic part plays a fundamental role as well. In fact: it is this aspect of the evolution that determines what states are available and their description. Suppose we study the motion of a cannonball, its state under gravitational and inertial forces will be properly described by the position and momentum of the center of mass. While light and air molecules scatter off its surface randomly, its trajectory is not greatly affect as it is a massive rigid body. Suppose we study the motion of a small particle, small enough that the random scattering does influence the trajectory, and it undergoes Brownian motion: its state will be a probability distribution for position and momentum of the center of mass. Gravitational and inertial forces have not changed, yet the set of states have. The issue is that the set of states must be closed under the non-deterministic evolution as well. That is: if we apply the non-deterministic evolution to any state, we still need to end up in one of the possible states. If the Brownian motion is strong enough, the state gets knocked out of the set of states determined by a simple position/momentum pair.

A similar more drastic effect: consider a book and its motion under gravitation and inertial forces, its state being the position and momentum of the center of mass. As we increase the temperature of the air around the book, its motion remains unaffected until, at some point, the book burns. Clearly, the non-deterministic evolution has brought one of the states outside the set, to the point that the system is no longer recognizable.

As we have seen, sometimes the state is identified by a distribution (either statistical or actual). Even in this case, the state can be deterministic and reversible. That is: given the distribution at one time we can determine the distribution at future times. The shape and the parameters of the distribution can be deterministic, even if the individual trajectories are not (those fall within the unstated part). In fact: we cannot assume trajectories and states are always defined at all for the unstated part. Consider a muon and its decay into an electron and two neutrinos: it is clear that the three outgoing particles have a state and trajectory of their own, it is clear that the resulting total mass and energy came from the muon. Yet, before the decay, we cannot ascribe an internal state and trajectory to those parts. The state of a muon is not some combination of the state of an electron and two neutrinos. That is: the unstated part is not just microstates.

Now that we have clarified what we mean by deterministic and reversible evolution (even on a statistical ensemble), and how also the non-deterministic part is key to define states (as the set as a whole needs to be invariant under it), we can ask ourselves: when is this assumption valid?

\begin{rationale}
The claim is that deterministic and reversible evolution is necessary to be able to study a physical system. That is: the only part of a system that can be studied through reliable and reproducible experimentation or observation is the part that undergoes deterministic and reversible evolution. We can provide different arguments that point in the same direction.

First, to be able to identify the system, we must first be able to tell it apart from the environment. Intuitively, we can distinguish between two chairs because we can move the first to another room and sit on it without having touched the second. We can manipulate the state of the first system without affecting the second, and vice-versa. So, to tell a system apart from its environment, there has to to be a certain amount of isolation between the two: if I affect its state, the state of other systems is not affected and vice-versa. This means that the system future and past states are only determined by its own state and the state undergoes deterministic and reversible evolution.

Second, the aim of physics is to write laws that can be used to make prediction that can be validated experimentally. If I drop an anvil from a tower, it will accelerate at $9.81 m/s^2$; if I want the anvil to reach the ground at $x m/s$ I have to drop it from $y m$. To the extent that we want to make predictions in time, we need to have a one-to-one correspondence between initial and final states.

Third, operationally we must reliably prepare and measure states. That is, we need a process for which the input settings of our preparing device determine the outgoing state of the system; and a process for which the incoming state of the system can be reconstructed by the output of the measuring device. That is, our system must participate in a deterministic and reversible process with the preparing and measuring device. How else could we calibrate our experimental apparatus?

This link between state definition and deterministic processes should not be too surprising as the state, in the context of thermodynamics and system theory, is often defined as \emph{the set of variables needed to determine the future evolution of the system}. As we saw before, this applies also to statistical processes: the distribution (the ensemble) as a whole can be indeed calculated, measured and prepared, not each individual element.

So, if it seems we need to require deterministic and reversible motion, why we call this an assumption? First of all, because it's an idealization: it can never be completely achieved in practice. A system can be prepared or measured up to a certain level of precision. Perfect isolation of a system is impossible both practically (e.g. black-body radiation, gravity, ...) and conceptually (e.g. if the system is perfectly isolated, we cannot interact with it: how can it studied through experimentation?). It's a simplifying assumption that can only be taken if the environment and the internal dynamics of the system interact in such a way that they little affect and are little affected by the aspect we are studying. As we saw before, for example, assuming that the states consists of the position and momentum of the center of mass means assuming that the Brownian motion of the body is negligible.

The second reason is that technically the assumption comes first. If we start looking for a law, we are implicitly assuming that there exists one to be found. The opposite assumption, there is no law to be found, can't really be experimentally proven exhaustively (i.e. you can only show that you have not found it among a certain class). Very often the hard part in science is finding for what system and in what conditions such assumption hold. Finding the law after that is a much simpler endeavour, and is something that is routinely done by students in their intro labs.
\end{rationale}

We are now ready to capture the elements of our discussion and their properties through mathematical definitions.

\begin{defn}\label{statedef}
The state space $\mathcal{S}$ of a physical system is a set.
\end{defn}

\begin{justification}
The collection of all possible states forms a set.
\end{justification}

\begin{defn}\label{detrevmap}
A deterministic and reversible evolution is a bijective map $f_{\Delta t}:\mathcal{S} \leftrightarrow \mathcal{S}$ on the state space.
\end{defn}

\begin{justification}
The system is deterministic: there exists a map $f:\mathcal{S} \rightarrow \mathcal{S}$ that returns the final state given the initial state. The system is reversible: there exists a map $g:\mathcal{S} \rightarrow \mathcal{S}$ that returns the initial state given the final state. $g \circ f = id$ as mapping forward and then backward must return the original element.
\end{justification}

%In practice, we use physical quantities to identify states. For example, the state for an ideal gas is given by pressure, volume and temperature. To clarify whether we are referring to a particular value of such quantity, or the quantity itself, we introduce the following terminology:
%\begin{center}
%    \begin{tabular}{ | p{2.5cm} | p{5.5cm} | }
%    \hline
%    State variable & A quantity (discrete or continuous) that must be specified to identify a state (e.g. position). \\ \hline
%    Label & A particular value for a state variable (e.g. position = $5m$). \\ \hline
%    Label range & A set of possible values for a state variable (e.g. position = $[4.5m, 5.5m]$). \\
%    \hline
%    \end{tabular}
%\end{center}

\begin{defn}\label{state_variable}
Let $\mathbbm{I}$ be a set. We call $q : \mathcal{S} \rightarrow \mathbbm{I}$ a \emph{state variable} and a \emph{possible value} (or \emph{possibility}) of $q$ an element $i \in \mathbbm{I}$. Also $\mathcal{s}_1 \neq \mathcal{s}_2 \Leftrightarrow \exists q : \mathcal{S} \rightarrow \mathbbm{I} | q(\mathcal{s}_1)\neq q(\mathcal{s}_2) \forall \mathcal{s}_1, \mathcal{s}_2 \in \mathcal{S}$.
\end{defn}

\begin{justification}
Consider a physical process that can distinguish between different initial conditions. Let $\mathbbm{I}$ be the set of all possible outcomes. To each possible state $\mathcal{s}$ will correspond one and only one outcome. Let $q : \mathcal{S} \rightarrow \mathbbm{I}$ be the map between state and outcome. Such object is physically well defined.

Let $\mathcal{s}_1, \mathcal{s}_2 \in \mathcal{S}$. Suppose $\mathcal{s}_1 \neq \mathcal{s}_2$. $\mathcal{s}_1$ and $\mathcal{s}_2$ are physically distinguishable. There exists a physical process with different outcomes. Let $q : \mathcal{S} \rightarrow \mathbbm{I}$ be the state variable associated to that process. Then $q(\mathcal{s}_1)\neq q(\mathcal{s}_2)$. Now suppose $\nexists q : \mathcal{S} \rightarrow \mathbbm{I} | q(\mathcal{s}_1) \neq q(\mathcal{s}_2)$. Then no process can distinguish between the two, they are physically indistinguishable therefore $\mathcal{s}_1 = \mathcal{s}_2$.
\end{justification}

\begin{defn}\label{independent_state_variables}
Let $q_1 : \mathcal{S} \rightarrow \mathbbm{I}_1$ and $q_2 : \mathcal{S} \rightarrow \mathbbm{I}_2$ be two state variables. They are said \emph{independent} if $q_1^{-1}(i_1)\cap q_2^{-1}(i_2) \neq 0 \forall i_1 \in \mathbbm{I}_1, i_2 \in \mathbbm{I}_2$.
\end{defn}

\begin{justification}
If $q_1$ are $q_2$ independent state variables we can modify a state such that only one variable is affected. TODO finish
\end{justification}

\begin{prop}\label{discrete_state_space}
The space state $\mathcal{S}$ for a system fully identified by a set of $n$ independent discrete state variables $q^i : \mathcal{S} \rightarrow \mathbbm{I}^i$ is isomorphic to $\prod\limits_{i=1}^n\mathbbm{I}^i$. Let $I^i \subseteq \mathbbm{I}^i$ be finite sets of possibilities for each variable. $\#(\bigcap\limits_{i=1}^{n}{q^i}^{-1}(I^i))=\prod\limits_{i=1}^{n}\#(I^i)$.
\end{prop}

\begin{proof}
TODO
\end{proof}

\begin{defn}\label{continuous_state_space}
The space state $\mathcal{Q}$ for a system fully identified by a set of $n$ independent continuous state variables is a manifold of dimension $n$.
\end{defn}

\begin{justification}
TODO: set of state variables locally define a chart; inherit topology from charts (small change in state makes small change in state variable, makes small changes in outcomes?)
\end{justification}


TODO: labels/state variables, cardinality of state variable, state id function (=1 for a particular state). Can we do for continuous variables as well? Can we find divergence free evolution for det/rev and independent variables?

\section{Composite systems and reducibility}

Given that scientific reductionism (i.e. the idea of reducing physical systems and interactions to the sum of their constituent parts in order to make them easier to study) is at the heart of fundamental physics, we should explore how to characterize a system in terms of its components. In general, this is quite a complicated thing to do, that requires intimate knowledge of the system at hand. So we simplify our problem and study an idealized case: one where the components are homogeneous and infinitesimal. What we'll find is that under the additional assumption that the system is reducible (i.e. its state is equivalent to the state of the parts) and that each part undergoes deterministic and reversible evolution, the motion is suitably described by the standard framework of classical Hamiltonian particle mechanics.

\subsection{Homogeneous systems}
As mentioned before, we will limit ourselves to the case where our system is homogeneous: the parts are indistinguishable, the set of states they can occupy is the same, they are under the same external and internal forces.\footnote{Non-homogeneous systems can in principle be decomposed in homogeneous components. Therefore their study can be reduced to the study of homogeneous systems.} What homogenous means depends on context (e.g. air can be thought as homogeneous if the mixture of gasses does not change in space or in time due to phase transitions or chemical processes) and one must check that such property is conserved by time evolution. Nonetheless, the state space of such system is richer in the sense that it comes equipped with the notion of composition and its rules. These are suitable captured mathematically by an additive abelian group.

\begin{defn}\label{reducible_state_space}
The state space $\mathcal{C}$ for a homogeneous decomposable system is an additive abelian (i.e. commutative) group.
\end{defn}

\begin{justification}
We claim $\mathcal{C}$ is an additive monoid. There exist a law of composition $+ : \mathcal{C} \times \mathcal{C} \rightarrow \mathcal{C}$ that takes two states and returns one that is the physical composition of the two. The domain and codomain match because the system is homogeneous. The law is commutative $\mathcal{c}_1 +\mathcal{c}_2 = \mathcal{c}_2+\mathcal{c}_1$ and associative $(\mathcal{c}_1 + \mathcal{c}_2) + \mathcal{c}_3 = \mathcal{c}_1 + (\mathcal{c}_2 + \mathcal{c}_3)$, as it does not matter in what order we physically compose the parts. There exist a unique zero element $\mathcal{c} + 0 = \mathcal{c}$ and it represent the physically empty state (i.e. no system).

We claim $\mathcal{C}$ is an additive group. As we want to describe changes during the evolution (i.e. $\hat{\mathcal{c}} = \mathcal{c} +\Delta \mathcal{c}$) we introduce an inverse $- : \mathcal{C} \rightarrow \mathcal{C}$ such that $\mathcal{c} + ( - \mathcal{c}) = 0 \forall \mathcal{c} \in \mathcal{C}$. Such inverse may introduce objects that do not properly represent a physical state, but a state change. A change of a physically distinguishable object is itself physically distinguishable therefore is physically well defined. We'll still call $\mathcal{C}$ the state space committing an abuse of terminology.
\end{justification}

\subsection{Classical homogeneous systems}
We also want to be able to express the state of the composite system in terms of the state of the parts. For example, given the state of a fluid we'll want to know how its parts are distributed. This in general will depend on how much the state of the composite system "knows" about its parts. For example, the position and orientation of an ideal rigid body is enough to define where all its constituents are, while the volume/pressure/temperature of an ideal gas is not enough to determine the position and momentum of all its molecules. Therefore we need to characterize the system further.

We will call a classical system one that is infinitesimally reducible. That is: it is made of arbitrarily small parts, which we call particles, and the state of the whole system is equivalent to the sum of the states of the parts. Given the state space $\mathcal{S}$ for the parts, identifying a composite state $\mathcal{c}$ is equivalent to quantify the amount of particles that are present in each state. Since the system is infinitesimally reducible, such amount can be increased or decreased infinitesimally and will be a real number. In other words: the classical state is a real valued distribution over $\mathcal{S}$. This intuitive picture is unfortunately not suitable to be generalized to different contexts, so we develop another.

Consider the following expression:
\begin{align*}
\mathcal{c} = \sum\limits_{i \in \mathbbm{I}} a_i \mathcal{e}(i)
\end{align*}
$\mathbbm{I}$ represents all the possible values of a state variable representing the best description of an infinitesimal part that a composite system can give. $\mathcal{e}(i)$ represents a composite system made of a unit amount of particles, all prepared in the state determined by $i$. $a_i$ represents a transformation that changes the whole state, without affecting the value $i$ for any infinitesimal part. For a classical system, $i$ represents the full information about the infinitesimal part, and therefore includes the value of all state variables; the only transformation left is increasing/decreasing the amount of particles by a scalar multiple. For a different system, where $i$ does not represent the full state of the parts, $\mathcal{e}(i)$ will be an ensemble and $a_i$ also represents an internal transformation of such ensable. This is intuitively how we decouple the composite state into the part $i$ that describes the infinitesimal parts, and the part $a_i$ that describes how the parts are composed. Giving a full description of the state space of a decomposable system means fully characterizing these components: the state variable $\mathbbm{I}$ and the set of transformations $A$. In the classical case $\mathcal{e}(i)$ form a basis of a real vector space, where $i$ is the set of state variable to identify the state of an an infinitesimal part, and $a_i$ a real valued coefficient that represents the amount of particles in each state, which describes how the parts are combined.

\begin{defn}\label{classical_vector space}
The state space $\mathcal{C}$ for a homogeneous classical (i.e. infinitesimally reducible) system is a vector space over $\mathbbm{R}$. It admits a basis isomorphic to the state space $\mathcal{S}$ of its particles (i.e. infinitesimal parts).
\end{defn}

\begin{justification}
The state space is an abelian group as the system is homogeneous and decomposable and \ref{reducible_state_space}.

We now consider the set of transformations $A$ that increase or decrease the amount of parts of the system by a constant. We claim $A$ is a field\footnote{Here field is intended in the abstract algebraic sense (a nonzero commutative division ring) which has no relationship to the field in the physics sense (a physical quantity with a value for each point in space).} isomorphic to $\mathbbm{R}$. Consider $a: \mathbbm{R} \rightarrow A$ the mapping between a number and the transformation that increases or decreases the size of the system by that number. This transformation exists: the system is infinitesimally decomposable and the amount can be changed continuously. Define on $A$ an addition $+: A \times A \rightarrow A$ and a multiplication $*: A \times A \rightarrow A$ such that $a(x) + a(y) = a(x+y)$ and $a(x) * a(y) = a(x*y)$, $x,y \in \mathbbm{R}$, so that the sum and product of the transformation is equal to the sum and product of their respective factors. $a$ is an isomorphism between $A$ and $\mathbbm{R}$ as fields.

We now claim that the state space $\mathcal{C}$ is a vector space over $\mathbbm{R}$. The abelian group $\mathcal{C}$ can be extended with the operations defined by $A$, as each element $a \in A$ is a map $a : \mathcal{C} \rightarrow \mathcal{C}$. The map has the following properties: $(a_1 + a_2) \mathcal{c} = a_1 \mathcal{c} + a_2 \mathcal{c}$, increasing the size of the system by the sum of two constant is the same as combining the separate increases, and $a (\mathcal{c}_1 + \mathcal{c}_2) = a \mathcal{c}_1 + a \mathcal{c}_2$, increasing the size of the total system is the same as the combination of the increased parts. $\mathcal{C}$ is a vector space over $A$, which is a field and isomorphic to $\mathbbm{R}$.

Consider now the injection map $\mathcal{e}: \mathcal{S} \hookrightarrow \mathcal{C}$ that for each particle state returns a composite state constituted only by a predetermined amount of particles in such state. The image of such injection is a basis for $\mathcal{C}$. We first claim that all elements are linearly independent. Consider $\sum\limits_{\mathcal{s}_i \in \mathcal{S}} a_i \mathcal{e}(\mathcal{s}_i)$, $a_i \in \mathbbm{R}$. This corresponds to a physical state with $a_i$ particles in each state $\mathcal{s}_i$. This is equal to the empty state if and only if $a_i=0 \forall i$. This holds even if $\mathcal{S}$ is not finite. We now claim they span. Consider $\mathcal{c} + \sum\limits_{\mathcal{s}_i \in \mathcal{S}} a_i \mathcal{e}(\mathcal{s}_i)$, $a_i \in \mathbbm{R}$. This can always be made equal to the empty state by setting $a_i$ to be the opposite of the amount of particles of $\mathcal{c}$ that are in state $\mathcal{s}_i$.
\end{justification}

\subsection{Classical hamiltonian assumption}


% Det/rev for infinitesimal part
% Inconsistent: 1. X=AxB A'=F(A) B'=F(B) Isolated sub-systems. Does not work for measuring. 2. A'=F(A) B'=F(A,B) Ideal non-interfering measurement: deterministic but not reversible. 3. A'=F(A,B) B'=F(A,B) X det/rev whole, but A not deterministic itself. But if you consider X a sub-system of X and something else, you can make the same argument. Whole universe deterministic, but none of its parts. 4. A=As x Au ; B=Bs x Bu ; As' = F(As) Au=F(X) Bu=F(X) Bs = F(As, Au) ; more realistic picture. Still non-reversible (if As' = F(As), then nothing else can be F(As) and have bijective map).

% Continuous state variable: Q is measured at constant t

We further characterize the relationship between the whole and the parts with the following assumption:

\begin{assump}[Classical hamiltonian assumption]\label{classical}
The evolution of the classical homogeneous system under study is deterministic, reversible and reducible.
\end{assump}

In other words: if we know the state and the evolution for the whole system, we also know the evolution for all its parts; if we know the state and evolution for all the parts, we know the state and the evolution for the whole system. By smaller we do not mean spatial extent, but of smaller amount (e.g. less mass): even at the same spatial location we can imagine that there are infinitely many components (e.g. the mass, as a number, is continuously divisible(.

\begin{rationale}
This is clearly a \emph{simplifying} assumption, and it is instructive to understand when it breaks down.

The first problem is methodological. As we saw before, we need access to a deterministic and reversible process to be able to study a system and define a state. For two balls, we can imagine to isolate some pieces (small enough to be considered infinitesimal in respect to the ball but big enough to contain enough molecules to be considered homogenous) and then describe the collision between the two by describing what happens at each piece. The classical assumption holds. For an electron and a photon, we cannot take pieces of the electron or the photon, study them in isolation and then describe how each part moves during Compton scattering. The classical assumption does not hold: we do not have suitable physical processes at our disposal.

The second problem is more conceptual. As we saw before, a system cannot be fully isolated from the environment. A certain amount of unstated part needs to remain so that the state of the system can be insulated from the external non-deterministic interaction. In practice, the state of each piece of our material is still given by the average of a collection of a large amount of molecules, their respective motion depending not just on the state of the system but on the external environment.  The classical assumption can therefore hold only if we assume that the description of each pieces is, again, not complete. Assuming it at a fundamental level would mean that the motion of each and every molecule of the system is only influenced by the state of the system, no external factor counts. As such, it would be a contradiction with it being a physical system: one we can study experimentally.

While ultimately flawed, the classical assumption can be considered valid for a great number of macroscopic system, and that is why is very useful. One should be cautioned, though, that nowhere in the assumption the spatial extent of the system is mentioned. We may as well have a macroscopic system where a clear independent state cannot be assigned to each part, and the assumption would not hold.
\end{rationale}

We are now ready to capture the elements of our discussion and their properties through mathematical definitions.


\section{Quantum systems}
%Note on composite systems and distinguishability. When putting together two classical system, just sum distributions. When putting together quantum system we have to clarify: combine to create one quantum system (sum in vector space) or two separate quantum system (symmetrized tensor product). In classical mechanics, 2a + 2b = (a+b) + (a+b). In quantum note |a> x |b> = |a>|b> + |b>|a> but (|a> + |b>) x (|a> + |b>) = |a>|a> + ... That is two quantum states in two different states is not the same as two quantum states spread equally over those two states. In classical they are.)


\begin{thebibliography}{0}

\bibitem{Shannon} Shannon, C. E., ``A mathematical theory of communication'', The Bell System Technical Journal, Vol. 27, pp. 379--423, 623--656, (1948).
\bibitem{Jaynes} Jaynes, E. T., ``Information theory and statistical mechanics'', Statistical Physics 3, pp. 181--218, (1963).
\bibitem{classical_dynamics} J. V. Jos\'{e}, E. J. Saletan, ``Classical Dynamics'', Cambridge University Press, (1998).
\bibitem{Gromov} Gromov, M. L., ``Pseudo holomorphic curves in symplectic manifolds'', Inventiones Mathematicae 82, pp. 307--347, (1985).
\bibitem{deGosson} de Gosson, M. A., ``The symplectic camel and the uncertainty principle: the tip of an iceberg?'', Foundations of Physics 39, pp. 194--214, (2009).
\bibitem{Stewart} Stewart, I., ``The symplectic camel'', Nature 329, pp. 17--18, (1987).
\bibitem{Lanczos} Lanczos, C., ``The variational principles of mechanics'', University of Toronto Press, (1949).
\bibitem{Synge} Synge, J. L., ``Classical dynamics'', Encyclopedia of Physics Vol 3/1, Springer (1960).
\bibitem{Struckmeier} Struckmeier, J., ``Hamiltonian dynamics on the symplectic extended phase space for autonomous and non-autonomous systems'', J. Phys. A: Math. Gen. 38, pp. 1257--1278, (2005).

\end{thebibliography}

\end{document}
