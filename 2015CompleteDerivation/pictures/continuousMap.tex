\documentclass{standalone}
\usepackage{tikz}
\usetikzlibrary{calc}
\begin{document}
	\begin{tikzpicture}[scale=.4]

		% INPUTS
				
		\newcommand{\displace}{23};			 % distance between the two plots, origin to origin		

		\newcommand{\axislength}{10};	   	 % axis lengeth
		\newcommand{\axisheight}{8};		 % axis height	

		\newcommand{\horlinelength}{7};		% length of all horizontal lines
		\newcommand{\vertlinelength}{7};	 % length of all vertical lines

		\newcommand{\lineheighta}{4};		  % y-coordinate of first horizontal line in both plots
		\newcommand{\lineheightb}{5.5};		 % y-coordinate of second horizontal line in both plots
		\newcommand{\functionheight}{2};  % y-coordinate of function in right plot
		
		\newcommand{\linea}{3};					  % x-coordinate of first vertical line in left plot
		\newcommand{\lineb}{5};				   	  % x-coordinate of second vertical line in left plot
		\newcommand{\linec}{4};					  % x-coordinate of vertical line in right plot	
		
		\newcommand{\hmargin}{2};
		\newcommand{\vmargin}{1.5};
		
		% styles		
	    \tikzstyle{axis}=[color=black];				
	    \tikzstyle{line}=[color=black,dashed];  % dashed better than dotted?
	    \tikzstyle{function}=[ultra thick,color=blue];			    

		% outer rectangle and name of diagram
		\draw ({-\hmargin - \axislength},{- \axisheight - \vmargin}) rectangle ({\displace + \axislength + \hmargin},{\axisheight + \vmargin});

	    				
		% axes for left plot
		\draw [style=axis] (0,-\axisheight) -- (0,\axisheight);		
		\draw [style=axis] (-\axislength,0) -- (\axislength,0);		
		
		% lines in left plot
		\draw [style=line] (\lineb,-\vertlinelength) -- (\lineb,\vertlinelength);
		\draw [style=line] (\linea,-\vertlinelength) -- (\linea,\vertlinelength);		
		
		\draw [style=line] (-\horlinelength,\lineheighta) -- (\horlinelength,\lineheighta);		
		\draw [style=line] (-\horlinelength,\lineheightb) -- (\horlinelength,\lineheightb);		

		% function in left plot		
		\draw [style=function] plot [smooth] coordinates {(-8,-3) (-4, -2) (0,0) (\linea,\lineheighta) (\lineb,\lineheightb) (8,6)};
		

		
		% axes for right plot
		\draw [style=axis] (\displace + 0,-\axisheight) -- (\displace + 0,\axisheight);		
		\draw [style=axis] (\displace + -\axislength,0) -- (\displace + \axislength,0);		
		
		% lines in right plot
		\draw [style=line] (\displace + \linec,-\vertlinelength) -- (\displace + \linec,\vertlinelength);

		\draw [style=line] (\displace + -\horlinelength,\lineheighta) -- (\displace + \horlinelength,\lineheighta);		
		\draw [style=line] (\displace + -\horlinelength,\lineheightb) -- (\displace + \horlinelength,\lineheightb);		

		% function including dot in right plot		
		\draw [style=function]	(\displace + -8,\functionheight) -- (\displace + 8,\functionheight);
		\fill [style=function] (\displace + \linec,4.75) circle (.25);
		\filldraw [draw=blue,fill=white,thick] (\displace + \linec,\functionheight) circle (.25);
		

	\end{tikzpicture}
	
\end{document}